
% Default to the notebook output style

    


% Inherit from the specified cell style.




    
\documentclass[11pt]{article}

    
    
    \usepackage[T1]{fontenc}
    % Nicer default font (+ math font) than Computer Modern for most use cases
    \usepackage{mathpazo}

    % Basic figure setup, for now with no caption control since it's done
    % automatically by Pandoc (which extracts ![](path) syntax from Markdown).
    \usepackage{graphicx}
    % We will generate all images so they have a width \maxwidth. This means
    % that they will get their normal width if they fit onto the page, but
    % are scaled down if they would overflow the margins.
    \makeatletter
    \def\maxwidth{\ifdim\Gin@nat@width>\linewidth\linewidth
    \else\Gin@nat@width\fi}
    \makeatother
    \let\Oldincludegraphics\includegraphics
    % Set max figure width to be 80% of text width, for now hardcoded.
    \renewcommand{\includegraphics}[1]{\Oldincludegraphics[width=.8\maxwidth]{#1}}
    % Ensure that by default, figures have no caption (until we provide a
    % proper Figure object with a Caption API and a way to capture that
    % in the conversion process - todo).
    \usepackage{caption}
    \DeclareCaptionLabelFormat{nolabel}{}
    \captionsetup{labelformat=nolabel}

    \usepackage{adjustbox} % Used to constrain images to a maximum size 
    \usepackage{xcolor} % Allow colors to be defined
    \usepackage{enumerate} % Needed for markdown enumerations to work
    \usepackage{geometry} % Used to adjust the document margins
    \usepackage{amsmath} % Equations
    \usepackage{amssymb} % Equations
    \usepackage{textcomp} % defines textquotesingle
    % Hack from http://tex.stackexchange.com/a/47451/13684:
    \AtBeginDocument{%
        \def\PYZsq{\textquotesingle}% Upright quotes in Pygmentized code
    }
    \usepackage{upquote} % Upright quotes for verbatim code
    \usepackage{eurosym} % defines \euro
    \usepackage[mathletters]{ucs} % Extended unicode (utf-8) support
    \usepackage[utf8x]{inputenc} % Allow utf-8 characters in the tex document
    \usepackage{fancyvrb} % verbatim replacement that allows latex
    \usepackage{grffile} % extends the file name processing of package graphics 
                         % to support a larger range 
    % The hyperref package gives us a pdf with properly built
    % internal navigation ('pdf bookmarks' for the table of contents,
    % internal cross-reference links, web links for URLs, etc.)
    \usepackage{hyperref}
    \usepackage{longtable} % longtable support required by pandoc >1.10
    \usepackage{booktabs}  % table support for pandoc > 1.12.2
    \usepackage[inline]{enumitem} % IRkernel/repr support (it uses the enumerate* environment)
    \usepackage[normalem]{ulem} % ulem is needed to support strikethroughs (\sout)
                                % normalem makes italics be italics, not underlines
    \usepackage{mathrsfs}
    

    
    
    % Colors for the hyperref package
    \definecolor{urlcolor}{rgb}{0,.145,.698}
    \definecolor{linkcolor}{rgb}{.71,0.21,0.01}
    \definecolor{citecolor}{rgb}{.12,.54,.11}

    % ANSI colors
    \definecolor{ansi-black}{HTML}{3E424D}
    \definecolor{ansi-black-intense}{HTML}{282C36}
    \definecolor{ansi-red}{HTML}{E75C58}
    \definecolor{ansi-red-intense}{HTML}{B22B31}
    \definecolor{ansi-green}{HTML}{00A250}
    \definecolor{ansi-green-intense}{HTML}{007427}
    \definecolor{ansi-yellow}{HTML}{DDB62B}
    \definecolor{ansi-yellow-intense}{HTML}{B27D12}
    \definecolor{ansi-blue}{HTML}{208FFB}
    \definecolor{ansi-blue-intense}{HTML}{0065CA}
    \definecolor{ansi-magenta}{HTML}{D160C4}
    \definecolor{ansi-magenta-intense}{HTML}{A03196}
    \definecolor{ansi-cyan}{HTML}{60C6C8}
    \definecolor{ansi-cyan-intense}{HTML}{258F8F}
    \definecolor{ansi-white}{HTML}{C5C1B4}
    \definecolor{ansi-white-intense}{HTML}{A1A6B2}
    \definecolor{ansi-default-inverse-fg}{HTML}{FFFFFF}
    \definecolor{ansi-default-inverse-bg}{HTML}{000000}

    % commands and environments needed by pandoc snippets
    % extracted from the output of `pandoc -s`
    \providecommand{\tightlist}{%
      \setlength{\itemsep}{0pt}\setlength{\parskip}{0pt}}
    \DefineVerbatimEnvironment{Highlighting}{Verbatim}{commandchars=\\\{\}}
    % Add ',fontsize=\small' for more characters per line
    \newenvironment{Shaded}{}{}
    \newcommand{\KeywordTok}[1]{\textcolor[rgb]{0.00,0.44,0.13}{\textbf{{#1}}}}
    \newcommand{\DataTypeTok}[1]{\textcolor[rgb]{0.56,0.13,0.00}{{#1}}}
    \newcommand{\DecValTok}[1]{\textcolor[rgb]{0.25,0.63,0.44}{{#1}}}
    \newcommand{\BaseNTok}[1]{\textcolor[rgb]{0.25,0.63,0.44}{{#1}}}
    \newcommand{\FloatTok}[1]{\textcolor[rgb]{0.25,0.63,0.44}{{#1}}}
    \newcommand{\CharTok}[1]{\textcolor[rgb]{0.25,0.44,0.63}{{#1}}}
    \newcommand{\StringTok}[1]{\textcolor[rgb]{0.25,0.44,0.63}{{#1}}}
    \newcommand{\CommentTok}[1]{\textcolor[rgb]{0.38,0.63,0.69}{\textit{{#1}}}}
    \newcommand{\OtherTok}[1]{\textcolor[rgb]{0.00,0.44,0.13}{{#1}}}
    \newcommand{\AlertTok}[1]{\textcolor[rgb]{1.00,0.00,0.00}{\textbf{{#1}}}}
    \newcommand{\FunctionTok}[1]{\textcolor[rgb]{0.02,0.16,0.49}{{#1}}}
    \newcommand{\RegionMarkerTok}[1]{{#1}}
    \newcommand{\ErrorTok}[1]{\textcolor[rgb]{1.00,0.00,0.00}{\textbf{{#1}}}}
    \newcommand{\NormalTok}[1]{{#1}}
    
    % Additional commands for more recent versions of Pandoc
    \newcommand{\ConstantTok}[1]{\textcolor[rgb]{0.53,0.00,0.00}{{#1}}}
    \newcommand{\SpecialCharTok}[1]{\textcolor[rgb]{0.25,0.44,0.63}{{#1}}}
    \newcommand{\VerbatimStringTok}[1]{\textcolor[rgb]{0.25,0.44,0.63}{{#1}}}
    \newcommand{\SpecialStringTok}[1]{\textcolor[rgb]{0.73,0.40,0.53}{{#1}}}
    \newcommand{\ImportTok}[1]{{#1}}
    \newcommand{\DocumentationTok}[1]{\textcolor[rgb]{0.73,0.13,0.13}{\textit{{#1}}}}
    \newcommand{\AnnotationTok}[1]{\textcolor[rgb]{0.38,0.63,0.69}{\textbf{\textit{{#1}}}}}
    \newcommand{\CommentVarTok}[1]{\textcolor[rgb]{0.38,0.63,0.69}{\textbf{\textit{{#1}}}}}
    \newcommand{\VariableTok}[1]{\textcolor[rgb]{0.10,0.09,0.49}{{#1}}}
    \newcommand{\ControlFlowTok}[1]{\textcolor[rgb]{0.00,0.44,0.13}{\textbf{{#1}}}}
    \newcommand{\OperatorTok}[1]{\textcolor[rgb]{0.40,0.40,0.40}{{#1}}}
    \newcommand{\BuiltInTok}[1]{{#1}}
    \newcommand{\ExtensionTok}[1]{{#1}}
    \newcommand{\PreprocessorTok}[1]{\textcolor[rgb]{0.74,0.48,0.00}{{#1}}}
    \newcommand{\AttributeTok}[1]{\textcolor[rgb]{0.49,0.56,0.16}{{#1}}}
    \newcommand{\InformationTok}[1]{\textcolor[rgb]{0.38,0.63,0.69}{\textbf{\textit{{#1}}}}}
    \newcommand{\WarningTok}[1]{\textcolor[rgb]{0.38,0.63,0.69}{\textbf{\textit{{#1}}}}}
    
    
    % Define a nice break command that doesn't care if a line doesn't already
    % exist.
    \def\br{\hspace*{\fill} \\* }
    % Math Jax compatibility definitions
    \def\gt{>}
    \def\lt{<}
    \let\Oldtex\TeX
    \let\Oldlatex\LaTeX
    \renewcommand{\TeX}{\textrm{\Oldtex}}
    \renewcommand{\LaTeX}{\textrm{\Oldlatex}}
    % Document parameters
    % Document title
    \title{TP03 - Fonctions}
    \author{Minh-Hoang DANG}
    
    
    
    

    % Pygments definitions
    
\makeatletter
\def\PY@reset{\let\PY@it=\relax \let\PY@bf=\relax%
    \let\PY@ul=\relax \let\PY@tc=\relax%
    \let\PY@bc=\relax \let\PY@ff=\relax}
\def\PY@tok#1{\csname PY@tok@#1\endcsname}
\def\PY@toks#1+{\ifx\relax#1\empty\else%
    \PY@tok{#1}\expandafter\PY@toks\fi}
\def\PY@do#1{\PY@bc{\PY@tc{\PY@ul{%
    \PY@it{\PY@bf{\PY@ff{#1}}}}}}}
\def\PY#1#2{\PY@reset\PY@toks#1+\relax+\PY@do{#2}}

\expandafter\def\csname PY@tok@w\endcsname{\def\PY@tc##1{\textcolor[rgb]{0.73,0.73,0.73}{##1}}}
\expandafter\def\csname PY@tok@c\endcsname{\let\PY@it=\textit\def\PY@tc##1{\textcolor[rgb]{0.25,0.50,0.50}{##1}}}
\expandafter\def\csname PY@tok@cp\endcsname{\def\PY@tc##1{\textcolor[rgb]{0.74,0.48,0.00}{##1}}}
\expandafter\def\csname PY@tok@k\endcsname{\let\PY@bf=\textbf\def\PY@tc##1{\textcolor[rgb]{0.00,0.50,0.00}{##1}}}
\expandafter\def\csname PY@tok@kp\endcsname{\def\PY@tc##1{\textcolor[rgb]{0.00,0.50,0.00}{##1}}}
\expandafter\def\csname PY@tok@kt\endcsname{\def\PY@tc##1{\textcolor[rgb]{0.69,0.00,0.25}{##1}}}
\expandafter\def\csname PY@tok@o\endcsname{\def\PY@tc##1{\textcolor[rgb]{0.40,0.40,0.40}{##1}}}
\expandafter\def\csname PY@tok@ow\endcsname{\let\PY@bf=\textbf\def\PY@tc##1{\textcolor[rgb]{0.67,0.13,1.00}{##1}}}
\expandafter\def\csname PY@tok@nb\endcsname{\def\PY@tc##1{\textcolor[rgb]{0.00,0.50,0.00}{##1}}}
\expandafter\def\csname PY@tok@nf\endcsname{\def\PY@tc##1{\textcolor[rgb]{0.00,0.00,1.00}{##1}}}
\expandafter\def\csname PY@tok@nc\endcsname{\let\PY@bf=\textbf\def\PY@tc##1{\textcolor[rgb]{0.00,0.00,1.00}{##1}}}
\expandafter\def\csname PY@tok@nn\endcsname{\let\PY@bf=\textbf\def\PY@tc##1{\textcolor[rgb]{0.00,0.00,1.00}{##1}}}
\expandafter\def\csname PY@tok@ne\endcsname{\let\PY@bf=\textbf\def\PY@tc##1{\textcolor[rgb]{0.82,0.25,0.23}{##1}}}
\expandafter\def\csname PY@tok@nv\endcsname{\def\PY@tc##1{\textcolor[rgb]{0.10,0.09,0.49}{##1}}}
\expandafter\def\csname PY@tok@no\endcsname{\def\PY@tc##1{\textcolor[rgb]{0.53,0.00,0.00}{##1}}}
\expandafter\def\csname PY@tok@nl\endcsname{\def\PY@tc##1{\textcolor[rgb]{0.63,0.63,0.00}{##1}}}
\expandafter\def\csname PY@tok@ni\endcsname{\let\PY@bf=\textbf\def\PY@tc##1{\textcolor[rgb]{0.60,0.60,0.60}{##1}}}
\expandafter\def\csname PY@tok@na\endcsname{\def\PY@tc##1{\textcolor[rgb]{0.49,0.56,0.16}{##1}}}
\expandafter\def\csname PY@tok@nt\endcsname{\let\PY@bf=\textbf\def\PY@tc##1{\textcolor[rgb]{0.00,0.50,0.00}{##1}}}
\expandafter\def\csname PY@tok@nd\endcsname{\def\PY@tc##1{\textcolor[rgb]{0.67,0.13,1.00}{##1}}}
\expandafter\def\csname PY@tok@s\endcsname{\def\PY@tc##1{\textcolor[rgb]{0.73,0.13,0.13}{##1}}}
\expandafter\def\csname PY@tok@sd\endcsname{\let\PY@it=\textit\def\PY@tc##1{\textcolor[rgb]{0.73,0.13,0.13}{##1}}}
\expandafter\def\csname PY@tok@si\endcsname{\let\PY@bf=\textbf\def\PY@tc##1{\textcolor[rgb]{0.73,0.40,0.53}{##1}}}
\expandafter\def\csname PY@tok@se\endcsname{\let\PY@bf=\textbf\def\PY@tc##1{\textcolor[rgb]{0.73,0.40,0.13}{##1}}}
\expandafter\def\csname PY@tok@sr\endcsname{\def\PY@tc##1{\textcolor[rgb]{0.73,0.40,0.53}{##1}}}
\expandafter\def\csname PY@tok@ss\endcsname{\def\PY@tc##1{\textcolor[rgb]{0.10,0.09,0.49}{##1}}}
\expandafter\def\csname PY@tok@sx\endcsname{\def\PY@tc##1{\textcolor[rgb]{0.00,0.50,0.00}{##1}}}
\expandafter\def\csname PY@tok@m\endcsname{\def\PY@tc##1{\textcolor[rgb]{0.40,0.40,0.40}{##1}}}
\expandafter\def\csname PY@tok@gh\endcsname{\let\PY@bf=\textbf\def\PY@tc##1{\textcolor[rgb]{0.00,0.00,0.50}{##1}}}
\expandafter\def\csname PY@tok@gu\endcsname{\let\PY@bf=\textbf\def\PY@tc##1{\textcolor[rgb]{0.50,0.00,0.50}{##1}}}
\expandafter\def\csname PY@tok@gd\endcsname{\def\PY@tc##1{\textcolor[rgb]{0.63,0.00,0.00}{##1}}}
\expandafter\def\csname PY@tok@gi\endcsname{\def\PY@tc##1{\textcolor[rgb]{0.00,0.63,0.00}{##1}}}
\expandafter\def\csname PY@tok@gr\endcsname{\def\PY@tc##1{\textcolor[rgb]{1.00,0.00,0.00}{##1}}}
\expandafter\def\csname PY@tok@ge\endcsname{\let\PY@it=\textit}
\expandafter\def\csname PY@tok@gs\endcsname{\let\PY@bf=\textbf}
\expandafter\def\csname PY@tok@gp\endcsname{\let\PY@bf=\textbf\def\PY@tc##1{\textcolor[rgb]{0.00,0.00,0.50}{##1}}}
\expandafter\def\csname PY@tok@go\endcsname{\def\PY@tc##1{\textcolor[rgb]{0.53,0.53,0.53}{##1}}}
\expandafter\def\csname PY@tok@gt\endcsname{\def\PY@tc##1{\textcolor[rgb]{0.00,0.27,0.87}{##1}}}
\expandafter\def\csname PY@tok@err\endcsname{\def\PY@bc##1{\setlength{\fboxsep}{0pt}\fcolorbox[rgb]{1.00,0.00,0.00}{1,1,1}{\strut ##1}}}
\expandafter\def\csname PY@tok@kc\endcsname{\let\PY@bf=\textbf\def\PY@tc##1{\textcolor[rgb]{0.00,0.50,0.00}{##1}}}
\expandafter\def\csname PY@tok@kd\endcsname{\let\PY@bf=\textbf\def\PY@tc##1{\textcolor[rgb]{0.00,0.50,0.00}{##1}}}
\expandafter\def\csname PY@tok@kn\endcsname{\let\PY@bf=\textbf\def\PY@tc##1{\textcolor[rgb]{0.00,0.50,0.00}{##1}}}
\expandafter\def\csname PY@tok@kr\endcsname{\let\PY@bf=\textbf\def\PY@tc##1{\textcolor[rgb]{0.00,0.50,0.00}{##1}}}
\expandafter\def\csname PY@tok@bp\endcsname{\def\PY@tc##1{\textcolor[rgb]{0.00,0.50,0.00}{##1}}}
\expandafter\def\csname PY@tok@fm\endcsname{\def\PY@tc##1{\textcolor[rgb]{0.00,0.00,1.00}{##1}}}
\expandafter\def\csname PY@tok@vc\endcsname{\def\PY@tc##1{\textcolor[rgb]{0.10,0.09,0.49}{##1}}}
\expandafter\def\csname PY@tok@vg\endcsname{\def\PY@tc##1{\textcolor[rgb]{0.10,0.09,0.49}{##1}}}
\expandafter\def\csname PY@tok@vi\endcsname{\def\PY@tc##1{\textcolor[rgb]{0.10,0.09,0.49}{##1}}}
\expandafter\def\csname PY@tok@vm\endcsname{\def\PY@tc##1{\textcolor[rgb]{0.10,0.09,0.49}{##1}}}
\expandafter\def\csname PY@tok@sa\endcsname{\def\PY@tc##1{\textcolor[rgb]{0.73,0.13,0.13}{##1}}}
\expandafter\def\csname PY@tok@sb\endcsname{\def\PY@tc##1{\textcolor[rgb]{0.73,0.13,0.13}{##1}}}
\expandafter\def\csname PY@tok@sc\endcsname{\def\PY@tc##1{\textcolor[rgb]{0.73,0.13,0.13}{##1}}}
\expandafter\def\csname PY@tok@dl\endcsname{\def\PY@tc##1{\textcolor[rgb]{0.73,0.13,0.13}{##1}}}
\expandafter\def\csname PY@tok@s2\endcsname{\def\PY@tc##1{\textcolor[rgb]{0.73,0.13,0.13}{##1}}}
\expandafter\def\csname PY@tok@sh\endcsname{\def\PY@tc##1{\textcolor[rgb]{0.73,0.13,0.13}{##1}}}
\expandafter\def\csname PY@tok@s1\endcsname{\def\PY@tc##1{\textcolor[rgb]{0.73,0.13,0.13}{##1}}}
\expandafter\def\csname PY@tok@mb\endcsname{\def\PY@tc##1{\textcolor[rgb]{0.40,0.40,0.40}{##1}}}
\expandafter\def\csname PY@tok@mf\endcsname{\def\PY@tc##1{\textcolor[rgb]{0.40,0.40,0.40}{##1}}}
\expandafter\def\csname PY@tok@mh\endcsname{\def\PY@tc##1{\textcolor[rgb]{0.40,0.40,0.40}{##1}}}
\expandafter\def\csname PY@tok@mi\endcsname{\def\PY@tc##1{\textcolor[rgb]{0.40,0.40,0.40}{##1}}}
\expandafter\def\csname PY@tok@il\endcsname{\def\PY@tc##1{\textcolor[rgb]{0.40,0.40,0.40}{##1}}}
\expandafter\def\csname PY@tok@mo\endcsname{\def\PY@tc##1{\textcolor[rgb]{0.40,0.40,0.40}{##1}}}
\expandafter\def\csname PY@tok@ch\endcsname{\let\PY@it=\textit\def\PY@tc##1{\textcolor[rgb]{0.25,0.50,0.50}{##1}}}
\expandafter\def\csname PY@tok@cm\endcsname{\let\PY@it=\textit\def\PY@tc##1{\textcolor[rgb]{0.25,0.50,0.50}{##1}}}
\expandafter\def\csname PY@tok@cpf\endcsname{\let\PY@it=\textit\def\PY@tc##1{\textcolor[rgb]{0.25,0.50,0.50}{##1}}}
\expandafter\def\csname PY@tok@c1\endcsname{\let\PY@it=\textit\def\PY@tc##1{\textcolor[rgb]{0.25,0.50,0.50}{##1}}}
\expandafter\def\csname PY@tok@cs\endcsname{\let\PY@it=\textit\def\PY@tc##1{\textcolor[rgb]{0.25,0.50,0.50}{##1}}}

\def\PYZbs{\char`\\}
\def\PYZus{\char`\_}
\def\PYZob{\char`\{}
\def\PYZcb{\char`\}}
\def\PYZca{\char`\^}
\def\PYZam{\char`\&}
\def\PYZlt{\char`\<}
\def\PYZgt{\char`\>}
\def\PYZsh{\char`\#}
\def\PYZpc{\char`\%}
\def\PYZdl{\char`\$}
\def\PYZhy{\char`\-}
\def\PYZsq{\char`\'}
\def\PYZdq{\char`\"}
\def\PYZti{\char`\~}
% for compatibility with earlier versions
\def\PYZat{@}
\def\PYZlb{[}
\def\PYZrb{]}
\makeatother


    % Exact colors from NB
    \definecolor{incolor}{rgb}{0.0, 0.0, 0.5}
    \definecolor{outcolor}{rgb}{0.545, 0.0, 0.0}



    
    % Prevent overflowing lines due to hard-to-break entities
    \sloppy 
    % Setup hyperref package
    \hypersetup{
      breaklinks=true,  % so long urls are correctly broken across lines
      colorlinks=true,
      urlcolor=urlcolor,
      linkcolor=linkcolor,
      citecolor=citecolor,
      }
    % Slightly bigger margins than the latex defaults
    
    \geometry{verbose,tmargin=1in,bmargin=1in,lmargin=1in,rmargin=1in}
    
    

    \begin{document}
    
    
    \maketitle
    
    

    
   \begin{Verbatim}[commandchars=\\\{\},fontsize=\scriptsize]
{\color{incolor}In [{\color{incolor}1}]:} \PY{c+c1}{\PYZhy{}\PYZhy{} connection: host=\PYZsq{}localhost\PYZsq{} dbname=\PYZsq{}TP03\PYZsq{} user=\PYZsq{}postgres\PYZsq{} password=\PYZsq{}postgres\PYZsq{}}
\end{Verbatim}

    \hypertarget{ecrire-une-fonction-qui-prend-deux-chauxeenes-et-retourne-la-longueur-de-la-plus-longue.}{%
\section{Ecrire une fonction qui prend deux chaînes et retourne la
longueur de la plus
longue.}\label{ecrire-une-fonction-qui-prend-deux-chauxeenes-et-retourne-la-longueur-de-la-plus-longue.}}

   \begin{Verbatim}[commandchars=\\\{\},fontsize=\scriptsize]
{\color{incolor}In [{\color{incolor}2}]:} \PY{k}{CREATE} \PY{k}{OR} \PY{k}{REPLACE} \PY{k}{FUNCTION} \PY{n}{longer\PYZus{}string}\PY{p}{(}\PY{n+nb}{text}\PY{p}{,} \PY{n+nb}{text}\PY{p}{)}
        \PY{k}{RETURNS} \PY{n+nb}{text} \PY{k}{AS} \PY{l+s}{\PYZdl{}}\PY{l+s}{\PYZdl{}}
            \PY{k}{BEGIN}
                \PY{k}{IF} \PY{n}{char\PYZus{}length}\PY{p}{(}\PY{n+nv}{\PYZdl{}1}\PY{p}{)} \PY{o}{\PYZgt{}} \PY{n}{char\PYZus{}length}\PY{p}{(}\PY{n+nv}{\PYZdl{}2}\PY{p}{)} \PY{k}{THEN}
                    \PY{k}{RETURN} \PY{n+nv}{\PYZdl{}1}\PY{p}{;}
                \PY{k}{ELSE}
                    \PY{k}{RETURN} \PY{n+nv}{\PYZdl{}2}\PY{p}{;}
                \PY{k}{END} \PY{k}{IF}\PY{p}{;}
            \PY{k}{END}\PY{p}{;}
        \PY{l+s}{\PYZdl{}}\PY{l+s}{\PYZdl{}} \PY{k}{LANGUAGE} \PY{n}{plpgsql}\PY{p}{;}
        
        \PY{k}{SELECT} \PY{o}{*} \PY{k}{FROM} \PY{n}{longer\PYZus{}string}\PY{p}{(}\PY{l+s+s1}{\PYZsq{}}\PY{l+s+s1}{court}\PY{l+s+s1}{\PYZsq{}}\PY{p}{,} \PY{l+s+s1}{\PYZsq{}}\PY{l+s+s1}{long}\PY{l+s+s1}{\PYZsq{}}\PY{p}{)}\PY{p}{;}
\end{Verbatim}

    \begin{Verbatim}[commandchars=\\\{\},fontsize=\footnotesize]
1 row(s) returned.

    \end{Verbatim}

    \begin{tabular}{l}
\toprule
 longer\_string   \\
\midrule
 court           \\
\bottomrule
\end{tabular}

    
    \hypertarget{ecrire-une-fonction-qui-prend-un-texte-et-affiche-notice-tous-les-mots-un-par-un.}{%
\section{Ecrire une fonction qui prend un texte et affiche (notice) tous
les mots un par
un.}\label{ecrire-une-fonction-qui-prend-un-texte-et-affiche-notice-tous-les-mots-un-par-un.}}

   \begin{Verbatim}[commandchars=\\\{\},fontsize=\scriptsize]
{\color{incolor}In [{\color{incolor}3}]:} \PY{k}{CREATE} \PY{k}{OR} \PY{k}{REPLACE} \PY{k}{FUNCTION} \PY{n}{afficher\PYZus{}mots}\PY{p}{(}\PY{n+nb}{text}\PY{p}{)}
        \PY{k}{RETURNS} \PY{n+nb}{void} \PY{k}{AS} \PY{l+s}{\PYZdl{}}\PY{l+s}{\PYZdl{}}
            \PY{k}{DECLARE}
                \PY{n}{iter} \PY{n+nb}{INTEGER} \PY{o}{:=} \PY{l+m+mf}{1}\PY{p}{;}
                \PY{n}{mot} \PY{n+nb}{TEXT} \PY{o}{:=} \PY{l+s+s1}{\PYZsq{}}\PY{l+s+s1}{\PYZsq{}}\PY{p}{;}
            \PY{k}{BEGIN}
                \PY{k}{LOOP}
                    \PY{n}{mot} \PY{o}{:=} \PY{n}{split\PYZus{}part}\PY{p}{(}\PY{n+nv}{\PYZdl{}1}\PY{p}{,} \PY{l+s+s1}{\PYZsq{}}\PY{l+s+s1}{ }\PY{l+s+s1}{\PYZsq{}}\PY{p}{,} \PY{n}{iter}\PY{p}{)}\PY{p}{;}
                    \PY{k}{RAISE} \PY{k}{NOTICE} \PY{l+s+s1}{\PYZsq{}}\PY{l+s+s1}{Mot \PYZpc{}: \PYZpc{}}\PY{l+s+s1}{\PYZsq{}}\PY{p}{,} \PY{n}{iter}\PY{p}{,} \PY{n}{mot}\PY{p}{;}
                    \PY{n}{iter} \PY{o}{:=} \PY{n}{iter} \PY{o}{+} \PY{l+m+mf}{1}\PY{p}{;}
                    \PY{k}{EXIT} \PY{k}{WHEN} \PY{n}{mot} \PY{o}{=} \PY{l+s+s1}{\PYZsq{}}\PY{l+s+s1}{\PYZsq{}}\PY{p}{;}
                \PY{k}{END} \PY{k}{LOOP}\PY{p}{;}
            \PY{k}{END}\PY{p}{;}
        \PY{l+s}{\PYZdl{}}\PY{l+s}{\PYZdl{}} \PY{k}{LANGUAGE} \PY{n}{plpgsql}\PY{p}{;}
        
        \PY{k}{SELECT} \PY{o}{*} \PY{k}{FROM} \PY{n}{afficher\PYZus{}mots}\PY{p}{(}\PY{l+s+s1}{\PYZsq{}}\PY{l+s+s1}{Dies irae, dies illa! Solvet saeclum in favilla, teste}
        \PY{l+s+s1}{David cum Sybilla!}\PY{l+s+s1}{\PYZsq{}}\PY{p}{)}
\end{Verbatim}

    \begin{Verbatim}[commandchars=\\\{\},fontsize=\footnotesize]
1 row(s) returned.
NOTICE:  Mot 1: Dies
NOTICE:  Mot 2: irae,
NOTICE:  Mot 3: dies
NOTICE:  Mot 4: illa!
NOTICE:  Mot 5: Solvet
NOTICE:  Mot 6: saeclum
NOTICE:  Mot 7: in
NOTICE:  Mot 8: favilla,
NOTICE:  Mot 9: teste
NOTICE:  Mot 10: David
NOTICE:  Mot 11: cum
NOTICE:  Mot 12: Sybilla!
NOTICE:  Mot 13:

    \end{Verbatim}

    \begin{tabular}{l}
\toprule
 afficher\_mots   \\
\midrule
                 \\
\bottomrule
\end{tabular}

    
    \hypertarget{ecrire-une-version-ituxe9rative-de-la-fonction-factorielle.}{%
\section{Ecrire une version itérative de la fonction
factorielle.}\label{ecrire-une-version-ituxe9rative-de-la-fonction-factorielle.}}

   \begin{Verbatim}[commandchars=\\\{\},fontsize=\scriptsize]
{\color{incolor}In [{\color{incolor}4}]:} \PY{k}{CREATE} \PY{k}{OR} \PY{k}{REPLACE} \PY{k}{FUNCTION} \PY{n}{factoriel}\PY{p}{(}\PY{n+nb}{integer}\PY{p}{)}
        \PY{k}{RETURNS} \PY{n+nb}{integer} \PY{k}{AS} \PY{l+s}{\PYZdl{}}\PY{l+s}{\PYZdl{}}
            \PY{k}{DECLARE}
                \PY{n}{res} \PY{n+nb}{integer} \PY{o}{:=} \PY{l+m+mf}{1}\PY{p}{;}
                \PY{n}{iter} \PY{n+nb}{integer} \PY{o}{:=} \PY{l+m+mf}{1}\PY{p}{;}
            \PY{k}{BEGIN}
                \PY{k}{LOOP}
                    \PY{n}{res} \PY{o}{:=} \PY{n}{res} \PY{o}{*} \PY{n}{iter}\PY{p}{;}
                    \PY{n}{iter} \PY{o}{:=} \PY{n}{iter} \PY{o}{+} \PY{l+m+mf}{1}\PY{p}{;}
                    \PY{k}{EXIT} \PY{k}{WHEN} \PY{n}{iter} \PY{o}{\PYZgt{}} \PY{n+nv}{\PYZdl{}1}\PY{p}{;}
                \PY{k}{END} \PY{k}{LOOP}\PY{p}{;}
                \PY{k}{RETURN} \PY{n}{res}\PY{p}{;}
            \PY{k}{END}\PY{p}{;}
        
        \PY{l+s}{\PYZdl{}}\PY{l+s}{\PYZdl{}} \PY{k}{LANGUAGE} \PY{n}{plpgsql}\PY{p}{;}
        
        \PY{k}{select} \PY{o}{*} \PY{k}{from} \PY{n}{factoriel}\PY{p}{(}\PY{l+m+mf}{3}\PY{p}{)}\PY{p}{;}
\end{Verbatim}

    \begin{Verbatim}[commandchars=\\\{\},fontsize=\footnotesize]
1 row(s) returned.

    \end{Verbatim}

    \begin{tabular}{r}
\toprule
   factoriel \\
\midrule
           6 \\
\bottomrule
\end{tabular}

    
    \hypertarget{calculer-les-deux-valeurs-suivantes}{%
\section{Calculer les deux valeurs suivantes
:}\label{calculer-les-deux-valeurs-suivantes}}

\hypertarget{somme-des-duruxe9es-de-tous-les-films-en-uxe9crivant-une-requuxeate-sql-puis-une-fonction-parcourant-tous-les-enregistrements-un-par-un-for-in.-assurez-vous-que-les-deux-muxe9thodes-retournent-le-muxeame-ruxe9sultat.}{%
\subsection{Somme des durées de tous les films en écrivant une requête
SQL puis une fonction parcourant tous les enregistrements un par un (for
in). Assurez-vous que les deux méthodes retournent le même
résultat.}\label{somme-des-duruxe9es-de-tous-les-films-en-uxe9crivant-une-requuxeate-sql-puis-une-fonction-parcourant-tous-les-enregistrements-un-par-un-for-in.-assurez-vous-que-les-deux-muxe9thodes-retournent-le-muxeame-ruxe9sultat.}}

\begin{itemize}
\tightlist
\item
  Requête SQL:
\end{itemize}

   \begin{Verbatim}[commandchars=\\\{\},fontsize=\scriptsize]
{\color{incolor}In [{\color{incolor}5}]:} \PY{k}{SELECT} \PY{n}{SUM}\PY{p}{(}\PY{n}{duree}\PY{p}{)} \PY{k}{FROM} \PY{n}{films}\PY{p}{;}
\end{Verbatim}

    \begin{Verbatim}[commandchars=\\\{\},fontsize=\footnotesize]
1 row(s) returned.

    \end{Verbatim}

    \begin{tabular}{r}
\toprule
   sum \\
\midrule
 10468 \\
\bottomrule
\end{tabular}

    
    \begin{itemize}
\tightlist
\item
  Fonction SQL:
\end{itemize}

   \begin{Verbatim}[commandchars=\\\{\},fontsize=\scriptsize]
{\color{incolor}In [{\color{incolor}6}]:} \PY{k}{CREATE} \PY{k}{OR} \PY{k}{REPLACE} \PY{k}{FUNCTION} \PY{n}{sum\PYZus{}duree\PYZus{}films}\PY{p}{(}\PY{p}{)}
        \PY{k}{RETURNS} \PY{n+nb}{integer} \PY{k}{AS} \PY{l+s}{\PYZdl{}}\PY{l+s}{\PYZdl{}}
            \PY{k}{DECLARE}
                \PY{n}{total\PYZus{}length} \PY{n+nb}{integer} \PY{o}{:=} \PY{l+m+mf}{0}\PY{p}{;}
                \PY{n}{unit\PYZus{}record} \PY{n+nb}{RECORD}\PY{p}{;}
            \PY{k}{BEGIN}
                \PY{k}{FOR} \PY{n}{unit\PYZus{}record} \PY{k}{IN} \PY{k}{SELECT} \PY{n}{duree} \PY{k}{FROM} \PY{n}{films} \PY{k}{LOOP}
                    \PY{n}{total\PYZus{}length} \PY{o}{:=} \PY{n}{total\PYZus{}length} \PY{o}{+} \PY{n}{unit\PYZus{}record}\PY{l+m+mf}{.}\PY{n}{duree}\PY{p}{;}
                \PY{k}{END} \PY{k}{LOOP}\PY{p}{;}
                \PY{k}{RETURN} \PY{n}{total\PYZus{}length}\PY{p}{;}
            \PY{k}{END}\PY{p}{;}
        \PY{l+s}{\PYZdl{}}\PY{l+s}{\PYZdl{}} \PY{k}{LANGUAGE} \PY{n}{plpgsql}\PY{p}{;}
        
        \PY{k}{SELECT} \PY{o}{*} \PY{k}{FROM} \PY{n}{sum\PYZus{}duree\PYZus{}films}\PY{p}{(}\PY{p}{)}\PY{p}{;}
\end{Verbatim}

    \begin{Verbatim}[commandchars=\\\{\},fontsize=\footnotesize]
1 row(s) returned.

    \end{Verbatim}

    \begin{tabular}{r}
\toprule
   sum\_duree\_films \\
\midrule
             10468 \\
\bottomrule
\end{tabular}

    
    \hypertarget{prix-moyen-dachat-des-films-en-uxe9crivant-une-requuxeate-sql-puis-une-fonction-parcourant-les-enregistrements-avec-un-curseur.-la-fonction-devra-lever-une-exception-sil-ny-a-aucun-uxe9luxe9ment-dans-la-table.-assurez-vous-que-les-deux-muxe9thodes-retournent-le-muxeame-ruxe9sultat-et-testez-lexception-en-vidant-la-table-truncate-table.}{%
\subsection{Prix moyen d'achat des films en écrivant une requête SQL
puis une fonction parcourant les enregistrements avec un curseur. La
fonction devra lever une exception s'il n'y a aucun élément dans la
table. Assurez-vous que les deux méthodes retournent le même résultat et
testez l'exception en vidant la table (TRUNCATE
TABLE).}\label{prix-moyen-dachat-des-films-en-uxe9crivant-une-requuxeate-sql-puis-une-fonction-parcourant-les-enregistrements-avec-un-curseur.-la-fonction-devra-lever-une-exception-sil-ny-a-aucun-uxe9luxe9ment-dans-la-table.-assurez-vous-que-les-deux-muxe9thodes-retournent-le-muxeame-ruxe9sultat-et-testez-lexception-en-vidant-la-table-truncate-table.}}

    \begin{itemize}
\tightlist
\item
  Requête SQL:
\end{itemize}

   \begin{Verbatim}[commandchars=\\\{\},fontsize=\scriptsize]
{\color{incolor}In [{\color{incolor}7}]:} \PY{k}{SELECT} \PY{n}{AVG}\PY{p}{(}\PY{n}{prixachat}\PY{p}{)} \PY{k}{FROM} \PY{n}{dvds}
\end{Verbatim}

    \begin{Verbatim}[commandchars=\\\{\},fontsize=\footnotesize]
1 row(s) returned.

    \end{Verbatim}

    \begin{tabular}{r}
\toprule
     avg \\
\midrule
 13.9561 \\
\bottomrule
\end{tabular}

    
    \begin{itemize}
\tightlist
\item
  Function SQL:
\end{itemize}

   \begin{Verbatim}[commandchars=\\\{\},fontsize=\scriptsize]
{\color{incolor}In [{\color{incolor}8}]:} \PY{k}{CREATE} \PY{k}{OR} \PY{k}{REPLACE} \PY{k}{FUNCTION} \PY{n}{avg\PYZus{}prix\PYZus{}achat}\PY{p}{(}\PY{p}{)}
        \PY{k}{RETURNS} \PY{k}{float} \PY{k}{AS} \PY{l+s}{\PYZdl{}}\PY{l+s}{\PYZdl{}}
            \PY{k}{DECLARE}
                \PY{n}{total} \PY{n+nb}{integer}\PY{p}{;}
                \PY{n}{unit\PYZus{}record} \PY{n+nb}{RECORD}\PY{p}{;}
                \PY{n}{total\PYZus{}prix} \PY{k}{float} \PY{o}{:=} \PY{l+m+mf}{0}\PY{p}{;}
            \PY{k}{BEGIN}
                \PY{k}{SELECT} \PY{n}{COUNT}\PY{p}{(}\PY{n}{nodvd}\PY{p}{)} \PY{k}{FROM} \PY{n}{dvds} \PY{k}{INTO} \PY{n}{total}\PY{p}{;}
                \PY{k}{IF} \PY{n}{total} \PY{o}{=} \PY{l+m+mf}{0} \PY{k}{THEN}
                    \PY{k}{RAISE} \PY{k}{EXCEPTION} \PY{l+s+s1}{\PYZsq{}}\PY{l+s+s1}{Table dvds est vide!}\PY{l+s+s1}{\PYZsq{}}
                    \PY{k}{USING} \PY{n}{HINT} \PY{o}{=} \PY{l+s+s1}{\PYZsq{}}\PY{l+s+s1}{Charger les données!}\PY{l+s+s1}{\PYZsq{}}\PY{p}{;}
                \PY{k}{END} \PY{k}{IF}\PY{p}{;}
        
                \PY{k}{FOR} \PY{n}{unit\PYZus{}record} \PY{k}{IN} \PY{k}{SELECT} \PY{n}{prixachat} \PY{k}{FROM} \PY{n}{dvds} \PY{k}{LOOP}
                    \PY{n}{total\PYZus{}prix} \PY{o}{:=} \PY{n}{total\PYZus{}prix} \PY{o}{+} \PY{n}{unit\PYZus{}record}\PY{l+m+mf}{.}\PY{n}{prixachat}\PY{p}{;}
                \PY{k}{END} \PY{k}{LOOP}\PY{p}{;}
                \PY{k}{RETURN} \PY{n}{total\PYZus{}prix}\PY{o}{/}\PY{n}{total}\PY{p}{;}
            \PY{k}{END}\PY{p}{;}
        \PY{l+s}{\PYZdl{}}\PY{l+s}{\PYZdl{}} \PY{k}{LANGUAGE} \PY{n}{plpgsql}\PY{p}{;}
        
        \PY{k}{SELECT} \PY{o}{*} \PY{k}{FROM} \PY{n}{avg\PYZus{}prix\PYZus{}achat}\PY{p}{(}\PY{p}{)}\PY{p}{;}
\end{Verbatim}

    \begin{Verbatim}[commandchars=\\\{\},fontsize=\footnotesize]
1 row(s) returned.

    \end{Verbatim}

    \begin{tabular}{r}
\toprule
   avg\_prix\_achat \\
\midrule
          13.9561 \\
\bottomrule
\end{tabular}

    
    \hypertarget{les-numuxe9ros-de-dvds-sont-compluxe8tement-chambouluxe9s.-ecrire-une-fonction-pour-le-renumuxe9roter-de-1-uxe0-n-de-maniuxe8re-consuxe9cutive.-faites-ce-quil-faut-pour-que-ne-pas-avoir-de-probluxe8mes-avec-les-cluxe9s-uxe9tranguxe8res.}{%
\section{Les numéros de dvds sont complètement chamboulés. Ecrire une
fonction pour le renuméroter de 1 à n de manière consécutive. Faites ce
qu'il faut pour que ne pas avoir de problèmes avec les clés
étrangères.}\label{les-numuxe9ros-de-dvds-sont-compluxe8tement-chambouluxe9s.-ecrire-une-fonction-pour-le-renumuxe9roter-de-1-uxe0-n-de-maniuxe8re-consuxe9cutive.-faites-ce-quil-faut-pour-que-ne-pas-avoir-de-probluxe8mes-avec-les-cluxe9s-uxe9tranguxe8res.}}

   \begin{Verbatim}[commandchars=\\\{\},fontsize=\scriptsize]
{\color{incolor}In [{\color{incolor}9}]:} \PY{k}{CREATE} \PY{k}{OR} \PY{k}{REPLACE} \PY{k}{FUNCTION} \PY{n}{corriger\PYZus{}nodvd}\PY{p}{(}\PY{p}{)}
        \PY{k}{RETURNS} \PY{n+nb}{void} \PY{k}{AS} \PY{l+s}{\PYZdl{}}\PY{l+s}{\PYZdl{}}
            \PY{k}{DECLARE}
                \PY{n}{unit\PYZus{}record} \PY{n+nb}{RECORD}\PY{p}{;}
                \PY{n}{iter} \PY{n+nb}{integer} \PY{o}{:=} \PY{l+m+mf}{1}\PY{p}{;}
                \PY{n}{current\PYZus{}nodvd} \PY{n+nb}{integer}\PY{p}{;}
            \PY{k}{BEGIN}
                \PY{k}{ALTER} \PY{k}{TABLE} \PY{n}{locations} \PY{k}{DROP} \PY{k}{CONSTRAINT} \PY{k}{IF} \PY{k}{EXISTS} \PY{n}{locations\PYZus{}nodvd\PYZus{}fkey}\PY{p}{;}
                \PY{k}{ALTER} \PY{k}{TABLE} \PY{n}{dvds} \PY{k}{DROP} \PY{k}{CONSTRAINT} \PY{k}{IF} \PY{k}{EXISTS} \PY{n}{dvds\PYZus{}pkey}\PY{p}{;}
        
                \PY{k}{FOR} \PY{n}{unit\PYZus{}record} \PY{k}{IN} \PY{k}{SELECT} \PY{n}{nodvd} \PY{k}{FROM} \PY{n}{dvds} \PY{k}{LOOP}
                    \PY{n}{current\PYZus{}nodvd} \PY{o}{:=} \PY{n}{unit\PYZus{}record}\PY{l+m+mf}{.}\PY{n}{nodvd}\PY{p}{;}
        
                    \PY{k}{IF} \PY{n}{iter} \PY{k}{IN} \PY{p}{(}\PY{k}{SELECT} \PY{n}{nodvd} \PY{k}{FROM} \PY{n}{dvds}\PY{p}{)} \PY{k}{THEN}
                        \PY{n}{iter} \PY{o}{:=} \PY{n}{iter} \PY{o}{+} \PY{l+m+mf}{1}\PY{p}{;}
                    \PY{k}{END} \PY{k}{IF}\PY{p}{;}
        
                    \PY{k}{UPDATE} \PY{n}{dvds} \PY{k}{SET} \PY{n}{nodvd} \PY{o}{=} \PY{n}{iter} \PY{k}{WHERE} \PY{n}{nodvd} \PY{o}{=} \PY{n}{current\PYZus{}nodvd}\PY{p}{;}
                    \PY{k}{UPDATE} \PY{n}{locations} \PY{k}{SET} \PY{n}{nodvd} \PY{o}{=} \PY{n}{iter} \PY{k}{WHERE} \PY{n}{nodvd} \PY{o}{=} \PY{n}{current\PYZus{}nodvd}\PY{p}{;}
        
                    \PY{n}{iter} \PY{o}{:=} \PY{n}{iter} \PY{o}{+} \PY{l+m+mf}{1}\PY{p}{;}
        
                \PY{k}{END} \PY{k}{LOOP}\PY{p}{;}
        
                \PY{k}{ALTER} \PY{k}{TABLE} \PY{n}{dvds} \PY{k}{ADD} \PY{k}{CONSTRAINT} \PY{n}{dvds\PYZus{}pkey} \PY{k}{PRIMARY} \PY{k}{KEY} \PY{p}{(}\PY{n}{nodvd}\PY{p}{)}\PY{p}{;}
        
                \PY{k}{ALTER} \PY{k}{TABLE} \PY{n}{locations} \PY{k}{ADD} \PY{k}{CONSTRAINT} \PY{n}{locations\PYZus{}nodvd\PYZus{}fkey} \PY{k}{FOREIGN} \PY{k}{KEY} \PY{p}{(}\PY{n}{nodvd}\PY{p}{)}
                \PY{k}{REFERENCES} \PY{n}{public}\PY{l+m+mf}{.}\PY{n}{dvds} \PY{p}{(}\PY{n}{nodvd}\PY{p}{)} \PY{k}{MATCH} \PY{k}{SIMPLE}
                \PY{k}{ON} \PY{k}{UPDATE} \PY{k}{CASCADE}
                \PY{k}{ON} \PY{k}{DELETE} \PY{k}{NO} \PY{k}{ACTION}\PY{p}{;}
            \PY{k}{END}\PY{p}{;}
        \PY{l+s}{\PYZdl{}}\PY{l+s}{\PYZdl{}} \PY{k}{LANGUAGE} \PY{n}{plpgsql}\PY{p}{;}
        
        \PY{k}{select} \PY{o}{*} \PY{k}{from} \PY{n}{corriger\PYZus{}nodvd}\PY{p}{(}\PY{p}{)}\PY{p}{;}
\end{Verbatim}

    \begin{Verbatim}[commandchars=\\\{\},fontsize=\footnotesize]
duplicate key value violates unique constraint "locations\_pkey"
DETAIL:  Key (nodvd, datelocation)=(6, 2015-03-21) already exists.
CONTEXT:  SQL statement "UPDATE locations SET nodvd = iter WHERE nodvd =
current\_nodvd"
PL/pgSQL function corriger\_nodvd() line 18 at SQL statement

    \end{Verbatim}

   \begin{Verbatim}[commandchars=\\\{\},fontsize=\scriptsize]
{\color{incolor}In [{\color{incolor}10}]:} \PY{k}{select} \PY{o}{*} \PY{k}{from} \PY{n}{dvds} \PY{k}{LIMIT} \PY{l+m+mf}{10}\PY{p}{;}
\end{Verbatim}

    \begin{Verbatim}[commandchars=\\\{\},fontsize=\footnotesize]
10 row(s) returned.

    \end{Verbatim}

    \begin{tabular}{rrl}
\toprule
   nodvd &   prixachat & titre                       \\
\midrule
       1 &          13 & Red Corner                  \\
       3 &          13 & There Goes the Neighborhood \\
       4 &          20 & Le dernier des Mohicans     \\
       5 &           9 & The Man from Earth          \\
       6 &          15 & Etre ou ne pas etre         \\
       7 &          17 & L amour en equation         \\
       8 &          20 & Flowers in the Attic        \\
       9 &          15 & Chinatown                   \\
      10 &          18 & Memento                     \\
      11 &          12 & Warrior                     \\
\bottomrule
\end{tabular}

    
    \hypertarget{ecrire-les-triggers-suivants}{%
\section{Ecrire les triggers suivants
:}\label{ecrire-les-triggers-suivants}}

\hypertarget{si-on-tente-dinsuxe9rer-ou-de-modifier-un-film-en-mettant-une-annuxe9e-infuxe9rieure-uxe0-1891-pourquoi-une-exception-est-levuxe9e.}{%
\subsection{Si on tente d'insérer ou de modifier un film en mettant une
année inférieure à 1891 (pourquoi ?) une exception est
levée.}\label{si-on-tente-dinsuxe9rer-ou-de-modifier-un-film-en-mettant-une-annuxe9e-infuxe9rieure-uxe0-1891-pourquoi-une-exception-est-levuxe9e.}}

En Mars 1891, William K. L. Dickson a développé le premier kinétoscope
en se basant sur les travaux de Louis Le Prince. Ici, on ne considère
que les films tourné avec un kinétoscope.

   \begin{Verbatim}[commandchars=\\\{\},fontsize=\scriptsize]
{\color{incolor}In [{\color{incolor}11}]:} \PY{k}{CREATE} \PY{k}{OR} \PY{k}{REPLACE} \PY{k}{FUNCTION} \PY{n}{verifier\PYZus{}anneesortie\PYZus{}film}\PY{p}{(}\PY{p}{)}
         \PY{k}{RETURNS} \PY{k}{TRIGGER} \PY{k}{AS} \PY{l+s}{\PYZdl{}}\PY{l+s}{\PYZdl{}}
             \PY{k}{BEGIN}
                 \PY{k}{RAISE} \PY{k}{EXCEPTION} \PY{l+s+s1}{\PYZsq{}}\PY{l+s+s1}{Année de la sortie de \PYZdq{}\PYZpc{}\PYZdq{} ne peut être avant 1891!, (input \PYZpc{})}\PY{l+s+s1}{\PYZsq{}}\PY{p}{,}
         \PY{n}{new}\PY{l+m+mf}{.}\PY{n}{titre}\PY{p}{,} \PY{n}{new}\PY{l+m+mf}{.}\PY{n}{anneesortie}\PY{p}{;}
             \PY{k}{END}\PY{p}{;}
         \PY{l+s}{\PYZdl{}}\PY{l+s}{\PYZdl{}} \PY{k}{LANGUAGE} \PY{n}{plpgsql}\PY{p}{;}
         
         \PY{k}{DROP} \PY{k}{TRIGGER} \PY{k}{IF} \PY{k}{EXISTS} \PY{n}{trigger\PYZus{}films\PYZus{}anneesortie} \PY{k}{on} \PY{n}{films}\PY{p}{;}
         \PY{k}{CREATE} \PY{k}{TRIGGER} \PY{n}{trigger\PYZus{}films\PYZus{}anneesortie}
             \PY{k}{BEFORE} \PY{k}{INSERT} \PY{k}{OR} \PY{k}{UPDATE} \PY{k}{OF} \PY{n}{anneesortie} \PY{k}{ON} \PY{n}{films}
             \PY{k}{FOR} \PY{k}{EACH} \PY{k}{ROW}
             \PY{k}{WHEN} \PY{p}{(}\PY{n}{NEW}\PY{l+m+mf}{.}\PY{n}{anneesortie} \PY{o}{\PYZlt{}} \PY{l+m+mf}{1891}\PY{p}{)}
                 \PY{k}{EXECUTE} \PY{k}{PROCEDURE} \PY{n}{verifier\PYZus{}anneesortie\PYZus{}film}\PY{p}{(}\PY{p}{)}\PY{p}{;}
\end{Verbatim}

   \begin{Verbatim}[commandchars=\\\{\},fontsize=\scriptsize]
{\color{incolor}In [{\color{incolor}12}]:} \PY{k}{DELETE} \PY{k}{FROM} \PY{n}{films} \PY{k}{WHERE} \PY{n}{titre} \PY{o}{=} \PY{l+s+s1}{\PYZsq{}}\PY{l+s+s1}{This Race Horse}\PY{l+s+s1}{\PYZsq{}}\PY{p}{;}
         \PY{k}{INSERT} \PY{k}{INTO} \PY{n}{films} \PY{k}{VALUES}\PY{p}{(}\PY{l+s+s1}{\PYZsq{}}\PY{l+s+s1}{This Race Horse}\PY{l+s+s1}{\PYZsq{}}\PY{p}{,} \PY{l+m+mf}{0.25}\PY{p}{,} \PY{l+s+s1}{\PYZsq{}}\PY{l+s+s1}{Cursus Limited}\PY{l+s+s1}{\PYZsq{}}\PY{p}{,} \PY{l+s+s1}{\PYZsq{}}\PY{l+s+s1}{E.Muybridge}\PY{l+s+s1}{\PYZsq{}}\PY{p}{,} \PY{l+m+mf}{1878}\PY{p}{,}
         \PY{l+s+s1}{\PYZsq{}}\PY{l+s+s1}{Drama}\PY{l+s+s1}{\PYZsq{}}\PY{p}{)}\PY{p}{;}
\end{Verbatim}

    \begin{Verbatim}[commandchars=\\\{\},fontsize=\footnotesize]
Année de la sortie de "This Race Horse" ne peut être avant 1891!, (input 1878)
CONTEXT:  PL/pgSQL function verifier\_anneesortie\_film() line 3 at RAISE

    \end{Verbatim}

   \begin{Verbatim}[commandchars=\\\{\},fontsize=\scriptsize]
{\color{incolor}In [{\color{incolor}13}]:} \PY{k}{UPDATE} \PY{n}{films} \PY{k}{SET} \PY{n}{anneesortie} \PY{o}{=} \PY{l+m+mf}{1890} \PY{k}{WHERE} \PY{n}{titre} \PY{o}{=} \PY{l+s+s1}{\PYZsq{}}\PY{l+s+s1}{Red Corner}\PY{l+s+s1}{\PYZsq{}}\PY{p}{;}
         \PY{k}{SELECT} \PY{o}{*} \PY{k}{FROM} \PY{n}{films} \PY{k}{WHERE} \PY{n}{titre} \PY{o}{=} \PY{l+s+s1}{\PYZsq{}}\PY{l+s+s1}{Red Corner}\PY{l+s+s1}{\PYZsq{}}\PY{p}{;}
\end{Verbatim}

    \begin{Verbatim}[commandchars=\\\{\},fontsize=\footnotesize]
Année de la sortie de "Red Corner" ne peut être avant 1891!, (input 1890)
CONTEXT:  PL/pgSQL function verifier\_anneesortie\_film() line 3 at RAISE

    \end{Verbatim}

    \hypertarget{si-un-dvd-est-insuxe9ruxe9-dans-la-table-sans-prix-dachat-indiquuxe9-par-exemple-insert-into-dvds-nodvd-titre-values-1001-8-mm-alors-le-prix-dachat-est-fixuxe9-au-prix-dachat-moyen-des-dvds-correspondant-au-muxeame-film.-par-contre-si-le-prix-est-indiquuxe9-il-ne-faut-rien-faire.}{%
\subsection{Si un dvd est inséré dans la table sans prix d'achat indiqué
(par exemple « INSERT INTO dvds (NoDVD, Titre) VALUES (1001, `8 mm'); »)
alors le prix d'achat est fixé au prix d'achat moyen des dvds
correspondant au même film. Par contre si le prix est indiqué il ne faut
rien
faire.}\label{si-un-dvd-est-insuxe9ruxe9-dans-la-table-sans-prix-dachat-indiquuxe9-par-exemple-insert-into-dvds-nodvd-titre-values-1001-8-mm-alors-le-prix-dachat-est-fixuxe9-au-prix-dachat-moyen-des-dvds-correspondant-au-muxeame-film.-par-contre-si-le-prix-est-indiquuxe9-il-ne-faut-rien-faire.}}

   \begin{Verbatim}[commandchars=\\\{\},fontsize=\scriptsize]
{\color{incolor}In [{\color{incolor}14}]:} \PY{k}{CREATE} \PY{k}{OR} \PY{k}{REPLACE} \PY{k}{FUNCTION} \PY{n}{verifier\PYZus{}prixachat\PYZus{}dvds}\PY{p}{(}\PY{p}{)}
         \PY{k}{RETURNS} \PY{k}{TRIGGER} \PY{k}{AS} \PY{l+s}{\PYZdl{}}\PY{l+s}{\PYZdl{}}
             \PY{k}{DECLARE}
                 \PY{n}{prixAvg} \PY{k}{float}\PY{p}{;}
             \PY{k}{BEGIN}
                 \PY{k}{SELECT} \PY{n}{AVG}\PY{p}{(}\PY{n}{prixachat}\PY{p}{)} \PY{k}{FROM} \PY{n}{dvds} \PY{k}{INTO} \PY{n}{NEW}\PY{l+m+mf}{.}\PY{n}{prixachat} \PY{p}{;}
                 \PY{k}{RETURN} \PY{n}{NEW}\PY{p}{;}
             \PY{k}{END}\PY{p}{;}
         \PY{l+s}{\PYZdl{}}\PY{l+s}{\PYZdl{}} \PY{k}{LANGUAGE} \PY{n}{plpgsql}\PY{p}{;}
         
         \PY{k}{DROP} \PY{k}{TRIGGER} \PY{k}{IF} \PY{k}{EXISTS} \PY{n}{trigger\PYZus{}dvds\PYZus{}prixachat} \PY{k}{on} \PY{n}{dvds}\PY{p}{;}
         \PY{k}{CREATE} \PY{k}{TRIGGER} \PY{n}{trigger\PYZus{}dvds\PYZus{}prixachat}
             \PY{k}{BEFORE} \PY{k}{INSERT} \PY{k}{ON} \PY{n}{dvds}
             \PY{k}{FOR} \PY{k}{EACH} \PY{k}{ROW}
             \PY{k}{WHEN} \PY{p}{(} \PY{n}{NEW}\PY{l+m+mf}{.}\PY{n}{prixachat} \PY{k}{IS} \PY{k}{NULL} \PY{p}{)}
                 \PY{k}{EXECUTE} \PY{k}{PROCEDURE} \PY{n}{verifier\PYZus{}prixachat\PYZus{}dvds}\PY{p}{(}\PY{p}{)}\PY{p}{;}
\end{Verbatim}

   \begin{Verbatim}[commandchars=\\\{\},fontsize=\scriptsize]
{\color{incolor}In [{\color{incolor}15}]:} \PY{k}{DELETE} \PY{k}{FROM} \PY{n}{dvds} \PY{k}{WHERE} \PY{n}{nodvd} \PY{o}{=} \PY{l+m+mf}{1003}\PY{p}{;}
         \PY{k}{DELETE} \PY{k}{FROM} \PY{n}{dvds} \PY{k}{WHERE} \PY{n}{nodvd} \PY{o}{=} \PY{l+m+mf}{1002}\PY{p}{;}
         
         \PY{k}{INSERT} \PY{k}{INTO} \PY{n}{dvds} \PY{p}{(}\PY{n}{nodvd}\PY{p}{,} \PY{n}{titre}\PY{p}{)} \PY{k}{VALUES}\PY{p}{(}\PY{l+m+mf}{1003}\PY{p}{,} \PY{l+s+s1}{\PYZsq{}}\PY{l+s+s1}{8 mm}\PY{l+s+s1}{\PYZsq{}}\PY{p}{)}\PY{p}{;}
         \PY{k}{INSERT} \PY{k}{INTO} \PY{n}{dvds} \PY{k}{VALUES}\PY{p}{(}\PY{l+m+mf}{1002}\PY{p}{,} \PY{l+m+mf}{7}\PY{p}{,} \PY{l+s+s1}{\PYZsq{}}\PY{l+s+s1}{8 mm}\PY{l+s+s1}{\PYZsq{}}\PY{p}{)}\PY{p}{;}
         
         \PY{k}{SELECT} \PY{o}{*} \PY{k}{FROM} \PY{n}{dvds} \PY{k}{WHERE} \PY{n}{nodvd} \PY{o}{=} \PY{l+m+mf}{1003} \PY{k}{or} \PY{n}{nodvd} \PY{o}{=} \PY{l+m+mf}{1002}\PY{p}{;}
\end{Verbatim}

    \begin{Verbatim}[commandchars=\\\{\},fontsize=\footnotesize]
2 row(s) returned.

    \end{Verbatim}

    \begin{tabular}{rrl}
\toprule
   nodvd &   prixachat & titre   \\
\midrule
    1003 &          14 & 8 mm    \\
    1002 &           7 & 8 mm    \\
\bottomrule
\end{tabular}

    
    \hypertarget{trouver-au-moins-deux-triggers-pertinents-suppluxe9mentaires-sur-cette-base-et-impluxe9mentez-les.}{%
\section{Trouver au moins deux triggers pertinents supplémentaires sur
cette base et
implémentez-les.}\label{trouver-au-moins-deux-triggers-pertinents-suppluxe9mentaires-sur-cette-base-et-impluxe9mentez-les.}}

    \hypertarget{il-ny-a-pas-de-cluxe9-uxe9tranger-pour-les-clients-pour-la-table-locations}{%
\subsection{Il n'y a pas de clé étranger pour les clients pour la table
locations}\label{il-ny-a-pas-de-cluxe9-uxe9tranger-pour-les-clients-pour-la-table-locations}}

   \begin{Verbatim}[commandchars=\\\{\},fontsize=\scriptsize]
{\color{incolor}In [{\color{incolor}16}]:} \PY{k}{CREATE} \PY{k}{OR} \PY{k}{REPLACE} \PY{k}{FUNCTION} \PY{n}{verifier\PYZus{}clients\PYZus{}locations}\PY{p}{(}\PY{p}{)}
         \PY{k}{RETURNS} \PY{k}{TRIGGER} \PY{k}{AS} \PY{l+s}{\PYZdl{}}\PY{l+s}{\PYZdl{}}
             \PY{k}{BEGIN}
                 \PY{k}{IF} \PY{p}{(}\PY{n}{NEW}\PY{l+m+mf}{.}\PY{n}{noclient} \PY{k}{IN} \PY{p}{(}\PY{k}{SELECT} \PY{n}{noclient} \PY{k}{FROM} \PY{n}{clients}\PY{p}{)}\PY{p}{)} \PY{k}{THEN}
                     \PY{k}{RETURN} \PY{n}{NEW}\PY{p}{;}
                 \PY{k}{ELSE}
                     \PY{k}{RAISE} \PY{k}{EXCEPTION} \PY{l+s+s1}{\PYZsq{}}\PY{l+s+s1}{Vérifiez que le client \PYZpc{} existe avant!}\PY{l+s+s1}{\PYZsq{}}\PY{p}{,} \PY{n}{NEW}\PY{l+m+mf}{.}\PY{n}{noclient}\PY{p}{;}
                     \PY{c+c1}{\PYZhy{}\PYZhy{}RETURN NULL;}
                 \PY{k}{END} \PY{k}{IF}\PY{p}{;}
             \PY{k}{END}\PY{p}{;}
         \PY{l+s}{\PYZdl{}}\PY{l+s}{\PYZdl{}} \PY{k}{LANGUAGE} \PY{n}{plpgsql}\PY{p}{;}
         
         \PY{k}{DROP} \PY{k}{TRIGGER} \PY{k}{IF} \PY{k}{EXISTS} \PY{n}{trigger\PYZus{}locations\PYZus{}clients} \PY{k}{on} \PY{n}{locations}\PY{p}{;}
         \PY{k}{CREATE} \PY{k}{TRIGGER} \PY{n}{trigger\PYZus{}locations\PYZus{}clients}
             \PY{k}{BEFORE} \PY{k}{INSERT} \PY{k}{ON} \PY{n}{locations}
             \PY{k}{FOR} \PY{k}{EACH} \PY{k}{ROW}
             \PY{k}{EXECUTE} \PY{k}{PROCEDURE} \PY{n}{verifier\PYZus{}clients\PYZus{}locations}\PY{p}{(}\PY{p}{)}\PY{p}{;}
\end{Verbatim}

   \begin{Verbatim}[commandchars=\\\{\},fontsize=\scriptsize]
{\color{incolor}In [{\color{incolor}17}]:} \PY{k}{INSERT} \PY{k}{INTO} \PY{n}{locations} \PY{k}{VALUES} \PY{p}{(}\PY{l+m+mf}{1003}\PY{p}{,} \PY{k}{current\PYZus{}date}\PY{p}{,} \PY{l+m+mf}{666}\PY{p}{,} \PY{l+m+mf}{3}\PY{p}{)}\PY{p}{;}
\end{Verbatim}

    \begin{Verbatim}[commandchars=\\\{\},fontsize=\footnotesize]
Vérifiez que le client 666 existe avant!
CONTEXT:  PL/pgSQL function verifier\_clients\_locations() line 6 at RAISE

    \end{Verbatim}

    \hypertarget{lors-de-linsertion-dans-la-table-location-si-la-date-nest-pas-pruxe9cisuxe9-on-met-la-date-du-jour}{%
\subsection{Lors de l'insertion dans la table location, si la date n'est
pas précisé, on met la date du
jour}\label{lors-de-linsertion-dans-la-table-location-si-la-date-nest-pas-pruxe9cisuxe9-on-met-la-date-du-jour}}

   \begin{Verbatim}[commandchars=\\\{\},fontsize=\scriptsize]
{\color{incolor}In [{\color{incolor}18}]:} \PY{k}{CREATE} \PY{k}{OR} \PY{k}{REPLACE} \PY{k}{FUNCTION} \PY{n}{verifier\PYZus{}date\PYZus{}locations}\PY{p}{(}\PY{p}{)}
         \PY{k}{RETURNS} \PY{k}{TRIGGER} \PY{k}{AS} \PY{l+s}{\PYZdl{}}\PY{l+s}{\PYZdl{}}
             \PY{k}{BEGIN}
                 \PY{k}{SELECT} \PY{k}{current\PYZus{}date} \PY{k}{INTO} \PY{n}{NEW}\PY{l+m+mf}{.}\PY{n}{datelocation}\PY{p}{;}
                 \PY{k}{RETURN} \PY{n}{NEW}\PY{p}{;}
             \PY{k}{END}\PY{p}{;}
         \PY{l+s}{\PYZdl{}}\PY{l+s}{\PYZdl{}} \PY{k}{LANGUAGE} \PY{n}{plpgsql}\PY{p}{;}
         
         \PY{k}{DROP} \PY{k}{TRIGGER} \PY{k}{IF} \PY{k}{EXISTS} \PY{n}{trigger\PYZus{}locations\PYZus{}date} \PY{k}{on} \PY{n}{locations}\PY{p}{;}
         \PY{k}{CREATE} \PY{k}{TRIGGER} \PY{n}{trigger\PYZus{}locations\PYZus{}date}
             \PY{k}{BEFORE} \PY{k}{INSERT} \PY{k}{ON} \PY{n}{locations}
             \PY{k}{FOR} \PY{k}{EACH} \PY{k}{ROW}
             \PY{k}{WHEN}\PY{p}{(}\PY{n}{NEW}\PY{l+m+mf}{.}\PY{n}{datelocation} \PY{k}{IS} \PY{k}{NULL}\PY{p}{)}
                 \PY{k}{EXECUTE} \PY{k}{PROCEDURE} \PY{n}{verifier\PYZus{}date\PYZus{}locations}\PY{p}{(}\PY{p}{)}\PY{p}{;}
\end{Verbatim}

   \begin{Verbatim}[commandchars=\\\{\},fontsize=\scriptsize]
{\color{incolor}In [{\color{incolor}19}]:} \PY{k}{INSERT} \PY{k}{INTO} \PY{n}{locations}\PY{p}{(}\PY{n}{nodvd}\PY{p}{,} \PY{n}{noclient}\PY{p}{,} \PY{n}{dureelocation}\PY{p}{)} \PY{k}{VALUES} \PY{p}{(}\PY{l+m+mf}{1003}\PY{p}{,} \PY{l+m+mf}{2}\PY{p}{,} \PY{l+m+mf}{3}\PY{p}{)}\PY{p}{;}
         \PY{k}{SELECT} \PY{o}{*} \PY{k}{FROM} \PY{n}{locations} \PY{k}{WHERE} \PY{n}{noclient} \PY{o}{=} \PY{l+m+mf}{2} \PY{k}{and} \PY{n}{nodvd} \PY{o}{=} \PY{l+m+mf}{1003}\PY{p}{;}
\end{Verbatim}

    \begin{Verbatim}[commandchars=\\\{\},fontsize=\footnotesize]
1 row(s) returned.

    \end{Verbatim}

    \begin{tabular}{rlrr}
\toprule
   nodvd & datelocation   &   noclient &   dureelocation \\
\midrule
    1003 & 2018-12-01     &          2 &               3 \\
\bottomrule
\end{tabular}

    

    % Add a bibliography block to the postdoc
    
    
    
    \end{document}
